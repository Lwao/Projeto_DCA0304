\documentclass[a4paper, 12pt, openright, oneside]{article}


\usepackage{graphicx}			% usar gráficos
\usepackage{multicol,lipsum}	%
\usepackage[utf8]{inputenc}		% Codificacao do documento (conversão automática dos acentos)
\usepackage{indentfirst}		% Indenta o primeiro parágrafo de cada seção.
\usepackage{color}				% Controle das cores
\usepackage{graphicx}			% Inclusão de gráficos
\usepackage{microtype} 			% para melhorias de justificação
\usepackage{natbib} 			% citar
\usepackage{float}				% números em ponto flutuante
\usepackage[brazil]{babel}		% língua brasileira
\usepackage{setspace}			% espaçamento entre linhas
\usepackage[brazilian,hyperpageref]{backref}	 % Paginas com as citações na bibl

\usepackage{amsmath, amsfonts, amssymb}
\usepackage[top=3cm, bottom=2cm, left=3cm, right=2cm]{geometry}

\parskip15pt					% distância entre parágrafos fixa
\parindent30pt					%identação fixa

% ---
% Pacotes de citações
% ---
\usepackage[brazilian,hyperpageref]{backref}	 % Paginas com as citações na bibl
\usepackage[alf]{abntex2cite}	% Citações padrão ABNT

% --- 
% CONFIGURAÇÕES DE PACOTES
% --- 

% ---
% Configurações do pacote backref
% Usado sem a opção hyperpageref de backref
\renewcommand{\backrefpagesname}{Citado na(s) página(s):~}
% Texto padrão antes do número das páginas
\renewcommand{\backref}{}
% Define os textos da citação
\renewcommand*{\backrefalt}[4]{
	\ifcase #1 %
		Nenhuma citação no texto.%
	\or
		Citado na página #2.%
	\else
		Citado #1 vezes nas páginas #2.%
	\fi}%
% ---

\begin{document}
%\maketitle
\onehalfspacing
\begin{titlepage}
	\begin{center}
	
	\begin{figure}[!ht]
	\centering
	\includegraphics[width=2cm]{./ufrn.jpg}
	\end{figure}
		Universidade Federal do Rio Grande do Norte\\
		Centro de Tecnologia\\
		Departamento de Engenharia da Computação e Automação\\
		DCA0304 -- Métodos Computacionais em Engenharia\\
		\vspace{15pt}
        \vspace{95pt}
        \textbf{\large{Comparação entre Julia e outras linguagens de programação na eficiência de execução do método de Newton-Raphson para solução de sistema de equações não-lineares}}\\
		\vspace{3,5cm}
	\end{center}
	
	\begin{flushright}
			\item André Rodrigues Bezerra Madruga \\
			Bruno Matias de Sousa \\
			José Ricardo Bezerra de Araújo \\
			Levy Gabriel da Silva Galvão \\
 	\end{flushright}
	\vspace{1cm}
	
	\begin{center}
		\vspace{\fill}
		Novembro\\2018
	\end{center}
\end{titlepage}
%%%%%%%%%%%%%%%%%%%%%%%%%%%%%%%%%%%%%%%%%%%%%%%%%%%%%%%%%%%

% % % % % % % % %FOLHA DE ROSTO % % % % % % % % % %



\begin{titlepage}
	\begin{center}
	
	\begin{figure}[!ht]
	\centering
	\includegraphics[width=2cm]{./ufrn.jpg}
	\end{figure}

		Universidade Federal do Rio Grande do Norte\\
		Centro de Tecnologia\\
		Departamento de Engenharia da Computação e Automação\\
		DCA0304 -- Métodos Computacionais em Engenharia\\
\vspace{15pt}
        
        \vspace{85pt}
        
		\textbf{\large{Comparação entre Julia e outras linguagens de programação na eficiência de execução do método de Newton-Raphson para solução de sistema de equações não-lineares}}\\
	%	\large{Modelo\\
     %   		Validação do modelo clássico}
			
	\end{center}
\vspace{1,5cm}
	
	\begin{flushright}

   \begin{list}{}{
      \setlength{\leftmargin}{4.5cm}
      \setlength{\rightmargin}{0cm}
      \setlength{\labelwidth}{0pt}
      \setlength{\labelsep}{\leftmargin}}

      \item Relatório técnico referente à execução prática dos métodos numéricos para a solução de sistemas de equações não-lineares realizado na disciplina de Métodos Computacionais em Engenharia, como requisito parcial para avaliação da terceira unidade da discplina antes mencionada.

      \begin{list}{}{
      \setlength{\leftmargin}{0cm}
      \setlength{\rightmargin}{0cm}
      \setlength{\labelwidth}{0pt}
      \setlength{\labelsep}{\leftmargin}}


            \item Orientador: Profº. Drº. Paulo Sergio da Motta Pires

      \end{list}
   \end{list}
\end{flushright}
\vspace{1cm}
\begin{center}
		\vspace{\fill}
		 Novembro\\2018
			\end{center}
\end{titlepage}
\newpage
% % % % % % % % % % % % % % % % % % % % % % % % % %
\newpage
\tableofcontents
\thispagestyle{empty}

\newpage
\pagenumbering{arabic}
% % % % % % % % % % % % % % % % % % % % % % % % % % %
\section*{Resumo}

Lorem ipsum dolor sit amet, consectetur adipiscing elit. Quisque et gravida mauris. Phasellus at ipsum in nisl iaculis consequat. Fusce vulputate nisl ipsum, quis egestas justo accumsan et. Morbi consequat tellus a eros eleifend congue. Aenean laoreet mattis nunc, at iaculis orci imperdiet in. Donec a diam in sem auctor fringilla. Aenean euismod odio vel arcu pretium, in vehicula urna ultricies. 

 \noindent
 \textbf{Palavras-chaves}: métodos. computacionais. engenharia.
\newpage

% ----------------------------------------------------------
% ELEMENTOS TEXTUAIS
% ----------------------------------------------------------

% ----------------------------------------------------------
% Introdução 
% ----------------------------------------------------------

\section{Introdução}


Muitos problemas da ciência da computação e de outras ciências podem serem abstraídos por meio de fórmulas e equações provenientes de uma linguagem matemática. Algumas dessas equações podem ser solucionadas de forma analítica utilizando a conceituação da literatura da área. Porém muitos outros problemas não possuem solução fechada por um método analítico, assim sendo necessário recorrer aos métodos iterativos, na maioria dos casos.

Ao longo dos anos os métodos iterativos solucionam problemas em diversas áreas, tais como: economia, engenharia, física, biologia, etc. Sua aplicação se baseia na aproximações para a solução do problema que melhoram em precisão de acordo com que aumentam o número de iterações. A extensa literatura na área de métodos iterativos só fortalece a importância de estudar a área.

A natureza repetitiva dos métodos iterativos sugere a sua execuação em recursos computacionais. Assim, a atividade "manufaturada" de realizar os cálculos é transferida para um computador, capaz de executá-las mais rapidamente.

Dessa forma, surgiu com o tempo uma tendência cada vez maior -- de acordo com que a tecnologia se desenvolvia -- de aliar a solução de problemas matemáticos aos métodos computacionais. Principalmente aqueles cuja solução analítica é difícil ou impossível. Um exemplo são os problemas de sistemas de equações não lineares que serão abordados no presente trabalho.

Para a obtenção das soluções desejadas foram utilizadas as linguagens de programação para realizar a comunicação entra a linguagem humana e matemática e a linguagem binária de máquina. Existem diversas linguagens com os mais diversos propósitos. Umas aplicadas ao gerenciamento de bancos de dados e outras com recursos dedicados aos métodos numéricos. 

Com a pluralidade de escolhas, basta ao profissional escolher aquela linguagem que mais se adequa às suas necessidades. O mais procurado nos dias atuais na solução de problemas por métodos numéricos é a linguagem que seja mais rápida, que utilize menos recurso computacional e ofereça a resposta mais precisa. 

Uma linguagem tida como forte candidata à preferida no cálculo numérico é a Julia. Uma linguagem bastante recente, cujo desenvolvimento começou em 2009 e teve a primeira versão de código aberto lançada em 2012.


\newpage

\section{Desenvolvimento}

Lorem ipsum dolor sit amet, consectetur adipiscing elit. Quisque et gravida mauris. Phasellus at ipsum in nisl iaculis consequat. Fusce vulputate nisl ipsum, quis egestas justo accumsan et. Morbi consequat tellus a eros eleifend congue. Aenean laoreet mattis nunc, at iaculis orci imperdiet in. Donec a diam in sem auctor fringilla. Aenean euismod odio vel arcu pretium, in vehicula urna ultricies.

Vivamus ut pharetra diam. Aliquam metus sem, tristique ac dignissim eu, pretium id velit. Donec id tincidunt odio. Fusce vehicula ac est quis convallis. Nullam sollicitudin euismod dolor, eget blandit turpis hendrerit at. In bibendum suscipit odio, at consequat erat laoreet id. Donec in tellus at nulla gravida egestas. Suspendisse non elementum leo. Nam viverra sapien sed velit tempor scelerisque. Nunc accumsan odio eget mi vehicula, vel interdum libero gravida. Pellentesque vitae molestie diam, quis vestibulum libero. Aenean finibus sapien diam, ac sagittis magna placerat nec. Ut maximus eros felis, vel egestas diam convallis sit amet. Integer interdum elementum turpis, sit amet luctus nisi scelerisque at. Donec eleifend arcu dictum metus ullamcorper tempor. Aenean placerat, arcu id suscipit vehicula, velit ipsum viverra lorem, et pretium velit tellus quis libero.

\subsection{Newton-Raphson para sistema de equações não-lineares}
Nunc accumsan odio eget mi vehicula, vel interdum libero gravida. Pellentesque vitae molestie diam, quis vestibulum libero. Aenean finibus sapien diam, ac sagittis magna placerat nec. Ut maximus eros felis, vel egestas diam convallis sit amet. Integer interdum elementum turpis, sit amet luctus nisi scelerisque at. Donec eleifend arcu dictum metus ullamcorper tempor. Aenean placerat, arcu id suscipit vehicula, velit ipsum viverra lorem, et pretium velit tellus quis libero.

\subsection{SENL propostos}
Nunc accumsan odio eget mi vehicula, vel interdum libero gravida. Pellentesque vitae molestie diam, quis vestibulum libero. Aenean finibus sapien diam, ac sagittis magna placerat nec. Ut maximus eros felis, vel egestas diam convallis sit amet. Integer interdum elementum turpis, sit amet luctus nisi scelerisque at. Donec eleifend arcu dictum metus ullamcorper tempor. Aenean placerat, arcu id suscipit vehicula, velit ipsum viverra lorem, et pretium velit tellus quis libero.

\subsection{Fluxograma do pseudocódigo}
Nunc accumsan odio eget mi vehicula, vel interdum libero gravida. Pellentesque vitae molestie diam, quis vestibulum libero. Aenean finibus sapien diam, ac sagittis magna placerat nec. Ut maximus eros felis, vel egestas diam convallis sit amet. Integer interdum elementum turpis, sit amet luctus nisi scelerisque at. Donec eleifend arcu dictum metus ullamcorper tempor. Aenean placerat, arcu id suscipit vehicula, velit ipsum viverra lorem, et pretium velit tellus quis libero.

\newpage

\section{Resultados}

Lorem ipsum dolor sit amet, consectetur adipiscing elit. Quisque et gravida mauris. Phasellus at ipsum in nisl iaculis consequat. Fusce vulputate nisl ipsum, quis egestas justo accumsan et. Morbi consequat tellus a eros eleifend congue. Aenean laoreet mattis nunc, at iaculis orci imperdiet in. Donec a diam in sem auctor fringilla. Aenean euismod odio vel arcu pretium, in vehicula urna ultricies.

Vivamus ut pharetra diam. Aliquam metus sem, tristique ac dignissim eu, pretium id velit. Donec id tincidunt odio. Fusce vehicula ac est quis convallis. Nullam sollicitudin euismod dolor, eget blandit turpis hendrerit at. In bibendum suscipit odio, at consequat erat laoreet id. Donec in tellus at nulla gravida egestas. Suspendisse non elementum leo. Nam viverra sapien sed velit tempor scelerisque. Nunc accumsan odio eget mi vehicula, vel interdum libero gravida. Pellentesque vitae molestie diam, quis vestibulum libero. Aenean finibus sapien diam, ac sagittis magna placerat nec. Ut maximus eros felis, vel egestas diam convallis sit amet. Integer interdum elementum turpis, sit amet luctus nisi scelerisque at. Donec eleifend arcu dictum metus ullamcorper tempor. Aenean placerat, arcu id suscipit vehicula, velit ipsum viverra lorem, et pretium velit tellus quis libero.

\subsection{Raízes}
Nunc accumsan odio eget mi vehicula, vel interdum libero gravida. Pellentesque vitae molestie diam, quis vestibulum libero. Aenean finibus sapien diam, ac sagittis magna placerat nec. Ut maximus eros felis, vel egestas diam convallis sit amet. Integer interdum elementum turpis, sit amet luctus nisi scelerisque at. Donec eleifend arcu dictum metus ullamcorper tempor. Aenean placerat, arcu id suscipit vehicula, velit ipsum viverra lorem, et pretium velit tellus quis libero.

\subsection{Eficiência na execução}
Nunc accumsan odio eget mi vehicula, vel interdum libero gravida. Pellentesque vitae molestie diam, quis vestibulum libero. Aenean finibus sapien diam, ac sagittis magna placerat nec. Ut maximus eros felis, vel egestas diam convallis sit amet. Integer interdum elementum turpis, sit amet luctus nisi scelerisque at. Donec eleifend arcu dictum metus ullamcorper tempor. Aenean placerat, arcu id suscipit vehicula, velit ipsum viverra lorem, et pretium velit tellus quis libero.

\newpage

\section{Conclusões}

Lorem ipsum dolor sit amet, consectetur adipiscing elit. Quisque et gravida mauris. Phasellus at ipsum in nisl iaculis consequat. Fusce vulputate nisl ipsum, quis egestas justo accumsan et. Morbi consequat tellus a eros eleifend congue. Aenean laoreet mattis nunc, at iaculis orci imperdiet in. Donec a diam in sem auctor fringilla. Aenean euismod odio vel arcu pretium, in vehicula urna ultricies.

Vivamus ut pharetra diam. Aliquam metus sem, tristique ac dignissim eu, pretium id velit. Donec id tincidunt odio. Fusce vehicula ac est quis convallis. Nullam sollicitudin euismod dolor, eget blandit turpis hendrerit at. In bibendum suscipit odio, at consequat erat laoreet id. Donec in tellus at nulla gravida egestas. Suspendisse non elementum leo. Nam viverra sapien sed velit tempor scelerisque. Nunc accumsan odio eget mi vehicula, vel interdum libero gravida. Pellentesque vitae molestie diam, quis vestibulum libero. Aenean finibus sapien diam, ac sagittis magna placerat nec. Ut maximus eros felis, vel egestas diam convallis sit amet. Integer interdum elementum turpis, sit amet luctus nisi scelerisque at. Donec eleifend arcu dictum metus ullamcorper tempor. Aenean placerat, arcu id suscipit vehicula, velit ipsum viverra lorem, et pretium velit tellus quis libero.

\newpage


% Referências bibliográficas

\nocite{associaccao1989nbr}
\nocite{marconi2003fundamentos}
\bibliographystyle{abbrv}
\bibliography{refs}

%\begin{figure}[!htb]
%\centering
%\includegraphics[height=6cm]{Photos/galvanometro.jpg}
%\caption{Na esquerda, um galvanômetro normalmente usado em circuitos, no centro, um galvanômetro com a estrutura iterna à amostra com fins didáticos e na direita as suas escalas. Fonte: própria.}
%\label{galvanometro}
%\end{figure}
\end{document}
